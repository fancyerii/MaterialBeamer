%!TEX program = xelatex
\documentclass{beamer}

\usepackage[english]{babel}
\usepackage{graphicx,hyperref,url, materialbeamer}
\usepackage{braket}
\usepackage{euler}
\usepackage{listings}

\lstdefinestyle{customsql}{
  belowcaptionskip=1\baselineskip,
  breaklines=true,
  xleftmargin=\parindent,
  language=SQL,
  showstringspaces=false,
  basicstyle=\footnotesize\ttfamily,
  keywordstyle=\bfseries\color{green!40!black},
  commentstyle=\itshape\color{purple!40!black},
  identifierstyle=\color{blue},
  stringstyle=\color{orange},
}
\lstset{escapechar=@,style=customsql}

\setlength{\parskip}{\baselineskip} 

\usefonttheme{professionalfonts} % using non standard fonts for beamer
%\usefonttheme{serif}

% The title of the presentation:
%  - first a short version which is visible at the bottom of each slide;
%  - second the full title shown on the title slide;
\title[Fb Tao]{Facebook Tao}

% Optional: a subtitle to be dispalyed on the title slide
\subtitle{Distributed Data Store for the Social Graph}

% The author(s) of the presentation:
%  - again first a short version to be displayed at the bottom;
%  - next the full list of authors, which may include contact information;
\author[L. Lancia \& G. Salillari]{
  L. Lancia \\ G. Salillari
  } 
  
%\titlegraphic{\includegraphics[width=\textwidth]{atac-logo}}

% The institute:
%  - to start the name of the university as displayed on the top of each slide
%    this can be adjusted such that you can also create a Dutch version
%  - next the institute information as displayed on the title slide
\institute[Sapienza Università di Roma]{
Cloud Computing\\
  Master Degree in Data Science \\
  Sapienza Università di Roma}

% Add a date and possibly the name of the event to the slides
%  - again first a short version to be shown at the bottom of each slide
%  - second the full date and event name for the title slide
\date[\today]{
 \today}




\providecommand{\di}{\mathop{}\!\mathrm{d}}
\providecommand*{\der}[3][]{\frac{d\if?#1?\else^{#1}\fi#2}{d #3\if?#1?\else^{#1}\fi}} 
 \providecommand*{\pder}[3][]{% 
    \frac{\partial\if?#1?\else^{#1}\fi#2}{\partial #3\if?#1?\else^{#1}\fi}% 
  }
\begin{document}

\begin{frame}
  \titlepage
\end{frame}

\begin{frame}
  \frametitle{Table of Contents}

  \tableofcontents
\end{frame}

%!TEX root = presentazionelancia.tex
\section{Introduction}
\begin{frame}[t]
\frametitle{Introduction}
What is TAO?

\begin{block}{Tao}
	is a geographically distribute store
	\begin{itemize}
		\item deployed at Facebook
	 	\item with efficient and timely access to social graph
	 	\item using a fixed set of query
	 	\item replacing memcache
	 	\item running on thousands of machines
	 	\item provide access to many PB of data
	 	\item process a billion reads ad millions of writes each second!
	 \end{itemize} 
\end{block}
\end{frame}

\begin{frame}
\frametitle{The social graph}
Facebook has more than 1 billion active user 
\begin{itemize}
	\item recording relationships,
	\item sharing interests,
	\item uploading pictures and \dots
\end{itemize}

The user experience of Fb comes from rapid, efficient and scalable access to the \emph{social graph}


    


\end{frame}


\end{document}
